\chapter{Einleitung}
\label{cha:Introduction}

\section{Motivation}
\textit{Natural Language Processing} ist die Möglichkeit der maschinellen Verarbeitung und Interpretation von natürlicher Sprache. Dabei beschränken sich die Anwendungen von \textit{NLP} nicht ausschließlich auf geschriebene Sprache in Form von Wörtern und Texten, sondern auch auf Lautsprache (gesprochene Sprache). Bereits seit mehreren Jahren wird im privaten Bereich auf sogenannte Sprachassistenten wie Cortana (Microsoft), Siri (Apple) und Amazon Echo (Amazon) gesetzt. Diese ermöglichen es, verschiedene Aufgaben mittels Lautsprache zu erledigen. Dabei sind Anwendungen wie z.B. das Ein und Ausschalten von Licht, das Abspielen von Playlists oder auch das Starten eines Anrufs nur einige der vielfältigen Anwendungsgebiete der Spracherkennung. Im Zusammenhang mit geschriebenen Wörtern und Texten wird \textit{NLP} vor allem bei der Interpretation von Texten eingesetzt. Ein Beispiel für die Anwendung von \textit{NLP} ist die Klassifizierung von Texten. Dies findet vor allem bei wissenschaftlichen Arbeiten Anwendung \cite{DannReconstructing}. Ein weiteres Anwendungsgebiet im Zusammenhang mit Texten ist das Zusammenfassen oder die semantische Überprüfung der Inhalte. Vor allem in den letzten Jahren hat das Thema Natural Language Processing im Zusammenhang mit sozialen Netzwerken an Bedeutung gewonnen. Eine Anwendung, welche in letzter Zeit sehr stark im Fokus von Behörden steht, ist die Erkennung von sogenannten \textit{Hasspostings} \cite{Nobata2016}. Durch die Masse an Postings die täglich in den sozialen Medien getätigt werden ist es nahezu unmöglich alle strafrechtlich relevanten Beiträge herauszufinden und diese gegebenenfalls zu löschen oder auch weitere rechtliche Schritte einzuleiten. Ein weiteres Anwendungsgebiet im Zusammenhang mit Postings in sozialen Medien ist das Analysieren dieser Postings zu Werbe- oder Marketingzwecken (User A postet, dass er gerne wieder auf Urlaub fahren würde, bekommt gezielt Werbung zu Urlaubsangeboten). Vor allem auf Grund des sehr werbelastigen Geschäftsmodells ist diese Analyse ein Thema von hoher Relevanz für die Betreiber von Werbeplatformen bzw. sozialen Medien wie \textit{Facebook} oder \textit{Twitter}.

Ein Gebiet, in dem die Möglichkeiten der Texterkennung sowie der Spracherkennung sehr stark zu einer Verbesserung beitragen könnten, sind IT-Prozesse mit einem starken Fokus auf zwischenmenschliche Kommunikation. In diesen kommunikationslastigen IT-Prozessen werden viele Daten in Form von menschlicher Sprache erzeugt. Diese kann entweder in Form von Texten (E-Mail, Dokumentation, Defect-Beschreibung), oder in Form von Sprache (Telefonanruf, Mailboxnachricht) vorliegen. Ein klassisches Beispiel für einen IT-Prozess bei dem die Daten in Form von menschlicher Sprache vorliegen sind Support-Prozesse. 

Ein konkretes Beispiel für die Vorselektierung bzw. Vorsortierung von Daten mit der Unterstützung von Natural Language Processing ist die Erkennung von Spam-Mails \cite{Dinsoreanu2014}. Beim Erkennen von Spam-Mails gibt es triviale Vorgehensweisen wie z.B. Blacklists mit Hilfe derer an Hand des Absenders festgestellt wird ob es sich um eine Spam-Mail handelt oder nicht. Komplexere Verfahren wären zum Beispiel die Analyse des Textes auf spezielle Schlagwörter oder auch den semantischen Inhalt des Textes mit Hilfe von Natural Language Processing.

Allgemein lässt sich sagen, dass der richtige und gezielte Einsatz von Natural Language Processing entscheidend zum Unternehmenserfolg beitragen kann. Vor allem in Prozessen, in denen die direkte Kommunikation zwischen Menschen eine wichtige Rolle spielt, kann durch die Verarbeitung und Auswertung der entstehenden Daten (textuell oder Sprache) eine Verbesserung erzielt werden.

Einige IT-Prozesse werden bereits rudimentär durch Natural Language Processing unterstützt, andere wiederum könnten von der Anwendung sehr profitieren. Ziel dieser Arbeit ist es, an Hand ausgewählter IT Prozesse (z.B. Support-Prozess, Anfrage eines Neukunden) beispielhaft zu erläutern, wie diese Prozesse durch die Anwendung von Natural Language Processing verbessert werden könnten. Der Fokus dieser Arbeit liegt dabei darauf, dass die Personen, die Teil dieses Prozesses sind, entlastet bzw. unterstützt werden. Dies kann die Mitarbeiter des Unternehmens betreffen, welche die Serviceleistung in Form eines Prozesses anbietet, aber auch die Kunden dieses Unternehmens durch eine möglicherweise schnellere Abwicklung dieses Prozesses (schneller bearbeitete Support-Anfragen, Chat-Bot statt FAQ) unterstützen und schlussendlich die Zufriedenheit auf beiden Seiten zu erhöhen. 

\section{Ziel und Methodik}
Der Inhalt dieser Arbeit besteht in erster Linie aus der Analyse ausgewählter IT-Prozesse und dem Erläutern der Möglichkeiten der Anwendung von Natural Language Processing zur Unterstützung dieser. 

Für die Analyse werden die IT-Prozesse eines klassischen Softwareentwicklungsbetriebes verwendet. Diese Prozesse werden kurz analysiert und stellen danach den IST Zustand dar. Es wird ein kurzer Überblick über den aktuellen Prozess gegeben und anschließend erläutert, welche Daten in diesem Prozess zwischen den einzelnen Schritten erzeugt bzw. verarbeitet werden. Danach wird gezeigt, wie mit Hilfe von Natural Language Processing eine Vorverarbeitung, Optimierung bzw. Erweiterung dieser Daten erfolgen kann. Am Ende sollte ein neuer Prozess skizziert werden, der durch die Unterstützung von Natural Language Processing verbessert wurde. 

Ein wichtiger Punkt, der bei der Analyse miteinbezogen wird, ist die Machbarkeit der einzelnen Verbesserungen, sowie der Limitierungen die sich mit der Anwendung ergeben könnten. Diese Limitierungen bzw. die Machbarkeit sollte in eine abschließende Bemerkung mit einfließen und Aufschluss darüber geben, unter welchen Voraussetzung \textit{NLP} eingesetzt werden könnte.
