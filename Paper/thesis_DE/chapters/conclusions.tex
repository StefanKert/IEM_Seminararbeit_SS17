\chapter{Fazit}
\label{cha:Conclusions}
Grundsätzlich lässt sich sagen, dass die Anwendung von \textit{Natural Language Processing} vor allem in Prozessen in denen viele Daten in Form von Text verarbeitet werden eine Verbesserung darstellen. Dies ist vor allem dort der Fall, wo eine Kommunikation mittels Chat, Email oder Onlineformular stattfindet. Für die meisten Anwendungen von \textit{NLP} ist eine manuelle Überprüfung durch einen Prozessteilnehmer oder eine Prozessteilnehmerin empfehlenswert, da die erhaltenen Ergebnisse häufig vom erwarteten Ergebnis abweichen können. Durch diese manuelle Überprüfung kann das System außerdem trainiert werden, sodass die gelieferten Ergebnisse in Zukunft möglicherweise besser dem Erwarteten entsprechen. 

Es wäre für Unternehmen von großem Vorteil, sich mit der Thematik des \textit{NLP} zu beschäftigen. Bereits rudimentäre Anwendungen, bringen Vorteile für die Prozesse und die Teilnehmer des Prozesses mit sich. Der Supportprozess ist hierfür ein gutes Beispiel. Bereits durch eine einfache Texterkennung kann eine Weiterleitung der Anfragen an die richtigen Stellen passieren. Der Aufwand für die Implementierung hält sich in Grenzen und erspart den Mitarbeitern die sich um die Anfragen kümmern häufig einiges an Arbeit.

Die skizzierten Verbesserungen sind dabei unabhängig von der Sprache bereits jetzt sehr gut einsetzbar, da kein aufwändige Syntaxanalyse des Textes im eigentlichen Sinn, sondern ein Vergleich des Inhaltes vorgenommen wird.

Für komplexere Anwendungsszenarien, wie dem Interpretieren eines Textes, gibt es vor allem im englischsprachigen Raum sehr stark Fortschritte. Im deutschen ist die Unterstützung teilweise nur rudimentär und nur in sehr speziellen Szenarien verwendbar. Einer der Hauptgründe für die bessere Unterstützung von englischer Sprache ist das wachsende Interesse von großen Unternehmen wie Microsoft, Apple oder Amazon an der Interpretation von Texten und Sprache im Allgemeinen und den Investitionen, die vor allem für die Unterstützung von englischer Sprache getätigt werden.
