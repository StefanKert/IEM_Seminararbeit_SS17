\chapter{Abstract}
As a result of the rise of big data applications and the need to make use of the collected data, the interest in machine learning applications has also grown considerably. Although this leads to more research on this topic and therefore positively affects both the speed and quality of the algorithms used, there is still high potential for improvements in both aspects, especially in the area of industrial applications with frequently changing data. 

Despite not being the newest research topic, the interest in domain adaptation is currently increasing -- among other things due to the latest developments in computing power. It is based on the idea of applying existing models to extended or modified application domains, instead of creating new ones for every use case. This topic is strongly related with the concept of transfer learning -- two terms that are not always clearly delimited from each other. \cite{Pan2010} \cite{Haslam2016}

To survey these terms, the content of this paper will focus on the description of domain adaptation, transfer learning and other related concepts on this field, like online and incremental learning. Aside from definitions and formalizations, the usage of adaptive learning methods in symbolic regression is the primary subject that will be discussed.


