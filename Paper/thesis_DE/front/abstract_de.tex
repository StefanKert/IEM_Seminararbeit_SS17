\chapter{Kurzfassung}
Die Anwendungen von \textit{Natural Language Processing} beschränken sich heutzutage nicht mehr ausschließlich auf Sprachassistenten, die dem Anwender oder der Anwenderin ermöglichen, über Spracheingabe verschiedene Aufgaben auszuführen. Es wurden in den letzten Jahren sehr viele weitere Aufgabengebiete erschlossen. Dazu gehören beispielsweise das Klassifizieren von Texten oder die Erkennung von Spam E-Mails \cite{DannReconstructing}\cite{Rohit2014}.

Dabei werden die Anwendungsmöglichkeiten von \textit{Natural Language Processing} von vielen Unternehmen unterschätzt. Zwar setzen immer mehr Unternehmen auf die Anwendung von \textit{Natural Language Processing} für die direkte Kommunikation mit Kunden und Kundinnen, häufig in Form von sogenannten Chat-Bots, investieren dann aber nicht in die Implementierung weiterer Anwendungen. 

Im Zuge dieser Arbeit, werden weitere Anwendungsmöglichkeiten von \textit{Natural Language Processing} im Unternehmensumfeld analysiert. Es wird dabei an Hand eines Beispielprozesses eines IT Unternehmens gezeigt, wie \textit{Natural Language Processing} integriert werden kann, welche Verbesserungen dadurch erwartet werden und ein Ausblick über die Machbarkeit dieser Integrierung gegeben.


