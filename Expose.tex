\documentclass[a4paper]{scrartcl}

\usepackage[T1]{fontenc}
\usepackage[utf8]{inputenc}
\usepackage[german]{babel} 

\addtokomafont{disposition}{\rmfamily}
\begin{document}
\title{Unterstützung von IT-Unternehmensprozessen durch verschiedene Anwendungen von Natural Language Processing}
\subtitle{Expose}
\author{Stefan Kert}
\date{\today}
\maketitle

\section{Inhalt}
Die Anwendungen rund um \textit{Natural Language Processing} haben in den letzten Jahren sehr stark an Bedeutung gewonnen. Sei es durch die klassischen Sprachassistenten wie Cortana (Microsoft), Siri (Apple) und Amazon Echo (Amazon), oder durch die vielseitigen Möglichkeiten der Analyse von Texten. Diese Analysen ermöglichen durch Natural Language Processing unter anderem eine Klassifizierung der Texte, aber auch das Zusammenfassen sowie eine semantische Überprüfung der Inhalte. Vor allem in den letzten Jahren hat das Thema Natural Language Processing im Zusammenhang mit sozialen Netzwerken an Bedeutung gewonnen. Hier ist vor allem die Erkennung von sogenannten \textit{Hasspostings} ein immer wiederkehrendes Thema \cite{Nobata2016}. 

Ein Gebiet, wo die Adaption dieser Anwendungen noch nicht sehr stark vorgenommen wurde sind Prozesse in Unternehmen. In den meisten IT-Prozessen sind auf irgendeine Art und Weise Menschen involviert und erzeugen Daten in Form von menschlicher Sprache. Ein klassisches Beispiel für einen IT-Prozessen bei dem die Daten in Form von menschlicher Sprache vorliegen sind Supportprozessen. Hier werden Anfragen gestellt, die meist in menschlicher Sprache verfasst sind. Schlussendlich werden diese Anfragen von Menschen verarbeitet, was häufig aber durch eine Vorselektierung oder automatische Klassifizierung der Anfragen vereinfacht werden könnte. Ein Beispiel für die Vorselektierung mit Hilfe von Natural Language Processing die in den meisten Unternehmen angewandt wird ist die Erkennung von Spam Emails \cite{Dinsoreanu2014}. Hier wird abhängig von zahlreichen Parametern wie Absender, Textinhalt oder auch semantischer Inhalt des Textes erkannt ob es sich um Spam oder um eine normale Email handelt.

Diese zahlreichen Anwendungen von Natural Language Processing sind sehr vielseitig und können auch entscheidend zum Unternehmenserfolg beitragen, insofern sie richtig eingesetzt werden.

\section{Motivation}
Wie bereits erwähnt gibt es in den verschiedenen IT-Prozessen zahlreiche Anwendungsmöglichkeiten für Natural Language Processing. Einige IT-Prozessen werden bereits rudimentär durch Natural Language Processing unterstützt andere wiederum könnten von der Anwendung sehr profitieren. Ziel dieser Arbeit ist es, an Hand einiger IT Prozesse beispielhaft zu erläutern, wie diese Prozesse durch die Anwendung von Natural Language Processing verbessert werden könnten. Vor allem in Bereichen in denen direkte Kommunikation zwischen Menschen vorhanden ist, kann durch die Verarbeitung der menschlichen Sprache, sei es in Form von Texten oder Sprache, eine Anwendung von Natural Language Processing erfolgen. Es geht dabei in erster Linie um die Entlastung bzw. Unterstützung der Mitarbeiter, welche Teil dieses Prozesses sind. Wie bereits erwähnt ist ein Beispiel für einen solchen Prozess die Support-anfragen von Kunden. Hier kann bereits vorweg eine Vorselektierung bzw. Klassifizierung der Anfrage erfolgen, sodass der Supportmitarbeiter oder die Supportmitarbeiterin bereits mehr Infos zu dem aktuellen Fall vorliegen hat. 

Da Natural Language Processing in sehr vielen Bereichen der IT eine Rolle spielt sollte auch die Verwendung bzw. die Unterstützung verschiedener IT-Prozesse immer mehr bedacht und auch implementiert werden, sodass diese schlussendlich zu einem noch besseren Unternehmenserfolg beitragen können.

\section{Forschungsfragen}
Die folgenden Fragen sollten im Zuge dieser Arbeit beantwortet werden:

\begin{enumerate}
	\item Natural Language Processing \cite{Jurafsky2008} \cite{Manning1999}
	\begin{enumerate}
			\item Was ist Natural Language Processing?
			\item Welche Anwendungen für Natural Language Processing gibt es?
			\item Wann sollte Natural Language Processing eingesetzt werden?
	\end{enumerate}
	\item Können IT-Prozesse durch die Unterstützung von Natural Language Processing verbessert werden? \cite{Huber2009}
	\begin{enumerate}
			\item In welchen IT-Prozessen kann Natural Language Processing Anwendung finden?
			\item Wie können diese IT-Prozesse verbessert werden?
	\end{enumerate}
\end{enumerate}

\section{Methodik}
Im Zuge dieser Arbeit werden verschiedene Möglichkeiten der Anwendung von Natural Language Processing erläutert und es sollte argumentiert werden, wie sich diese Anwendungen auf ausgewählte IT-Prozesse unterstützend auswirken. Dafür werden einige Prozesse ausgewählt und der IST Zustand analysiert. Es wird erläutert, wie dieser Prozess momentan funktioniert, welche Daten dieser Prozess erzeugt bzw. verlangt und wie sich dieser Prozess auf das Unternehmen auswirkt. Danach wird versucht durch die Anwendung von Natural Language Processing einige Prozessschritte zu optimieren bzw. den Prozess so zu erweitern, dass es den Teilnehmern und Teilnehmerinnen dieses Prozesses erleichtert wird das gewünschte Ergebnis zu erzielen.

\section{Inhaltsverzeichnis}
\begin{enumerate}
	\item Kurzfassung
	\item Einführung
	\begin{enumerate}
			\item Motivation
			\item Ziele
	\end{enumerate}
	\item Natural Language Processing
	\begin{enumerate}
			\item Übersicht
			\item Anwendungen
	\end{enumerate}
	\item Unterstützung ausgewählter IT-Prozessen durch die Anwendung von Natural Language Processing
	\begin{enumerate}
			\item Allgemeines
			\item Supportanfragen per Mail verarbeiten
			\item Anfragen von Neukunden
	\end{enumerate}
	\item Fazit
\end{enumerate}

\bibliographystyle{plain}
\bibliography{bibliography} 

\end{document}

